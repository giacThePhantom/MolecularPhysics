\chapter{Multidimensional integrals}
When working in multiple dimensions, the domain can have a shape that can produce effects on the integration procedure.
Another difficulty introduced by solving integrals in many dimensions is the choice of the order of integration.

\section{Definition}
The integral of a function $f(x,y)$ in two dimensions is the volume under the surface $z = f(x,y)$.
Supposing that the function is defined over a rectangular domain $(a,b)\times (c,d)$, the domain can be divided in many smaller rectangles with dimension $\Delta x\times \Delta y$.
The sum of all these subdomains covers the whole original domain.
In each subdomain we can define the infimum and the supremum.
The infimum of the function is the greatest element of $\mathbb{R}$ that is less than or equal to all elements of the function in the corresponding subdomain.
The supremum is the least element of $\mathbb{R}$ that is greater than or equal to all elements of the function in the corresponding subdomain.
A specific division of the domain is denoted as $P$.
The Darboux sums are defined as:

$$L(f,P)= \sum\limits_{\Delta x_k\times \Delta y_k}\sup\limits_{[\Delta x_k\times \Delta y_k]} f(x,y)(\Delta x_k\times \Delta y_k)$$

$$U(f,P)= \sum\limits_{\Delta x_k\times \Delta y_k}\inf\limits_{[\Delta x_k\times \Delta y_k]} f(x,y)(\Delta x_k\times \Delta y_k)$$

Where the sum is over all the subdomains labelled by $k$.
The function $f(x,y)$ is Riemann-integrable if:

$$\sup L(f,P) = \inf U(f,P) = I$$

when varying the partition; furthermore we can write that:

$$I = \int\limits_c^d\int\limits_a^bf(x,y)dxdy$$

This relation implies several properties.

\section{Properties}

	\subsection{Integrability}
	Let $f(x,y)$ be a continuous function from $(a,b)\times (c,d)$, then the function is integrable.
	This lemma states the integrability requirements for a small subset of functions, but the result can be extended to different kind of integrals and to more than two dimensions.

	\subsection{Order of integration}
	If $f(x,y)$ is continuous on $(a,b)\times (c,d)$ then:

	$$\int\limits_c^d\biggl(\int\limits_a^b f(x,y)dx\biggr)dy = \int\limits_a^b\biggl(\int\limits_c^d f(x,y)dy\biggr)dx$$

	Basically you can choose the order of integration as you see fit.

	\subsection{x-simple and y-simple domains}
	In some cases the domain of the function is not defined over a rectangular domain.
	One case of easy integration is when the domain is x-simple or y-simple.
	In the case of a y-simple domain, the function is bounded on the $x$ axis by two numerical values and on the $y$ axis by two continuous function $y = g_1(x)$ and $y = g_2(x)$.
	The case of a x-simple domain is the symmetric of the y-simple one.
	Let $f(x,y)$ be a continuous function defined on an x-simple domain $\Omega$:

	$$\Omega = \{(x,y)\in\mathbb{R}, c\le y\le d, h_1(y)\le x\le h_2(y)\}$$

	The integral can be computed as:

	$$\iint\limits_{\Omega} f = \int\limits_c^ddy\biggl(\int\limits_{h_1(y)}^{h_2(y)}f(x,y)dx\biggr)$$

	In the same way for a y-simple domain:

	$$\Omega = \{(x,y)\in\mathbb{R}, g_1(x)\le y\le g_2(x), a\le x\le b\}$$

	$$\iint\limits_\Omega f = \int\limits_a^bdx\biggl(\int\limits_{g_1(x)}^{g_2(x)}f(x,y)dy\biggr)$$

	\subsection{Change of variables}
	The absolute value of the jacobian determinant gives the change in the volume element when passing from a set of coordinates to another.
	In two dimension, given $f(x,y)$ defined on $\Omega$ and supposing to change the integral variables from $x$ and $y$ to $u$ and $v$:

	$$\iint\limits_\Omega f(x,y)dxdy = \iint\limits_{\Omega^*} f(x(u,v), y(u,v))|J|dudv$$

	Where $\Omega^*$ is the new region of integration in the $(u,v)$ plane.
