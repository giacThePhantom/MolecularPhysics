\chapter{Multidimensional integrals}
In many dimension the domain can have a shape that can produce effects on the integration procedure.
Another difficulty introduced by solving integrals in many dimensions is the choice of order of integrations.

\section{Definition}
The integral of a function $f(x,y)$ in two dimension is the volume under the surface $z = f(x,y)$.
Supposing that the function is defined over a rectangular domain $(a,b)\times (c,d)$ the domain can be divided in many smaller rectangles with dimension $\Delta x\times \Delta y$.
These subdomains cover the whole original domain.
In each subdomain the infimum and the supremum.
The infimum of the function is the greatest element of $\mathbb{R}$ that is less than or equal to all elements of the function on the corresponding subdomain.
The supremum is the least element of $\mathbb{R}$ that is less than or equal to all elements of the function on the corresponding subdomain.
A specific division of the domain is denoted as $P$.
The Darboux sums are defined as:

$$L(f,P)= \sum\limits_{\Delta x_k\times \Delta y_k}\sup\limits_{[\Delta x_k\times \Delta y_k]} f(x,y)(\Delta x_k\times \Delta y_k)$$

$$U(f,P)= \sum\limits_{\Delta x_k\times \Delta y_k}\inf\limits_{[\Delta x_k\times \Delta y_k]} f(x,y)(\Delta x_k\times \Delta y_k)$$

Where the sum is over all the subdomain labelled by $k$.
The function $f(x,y)$ is Riemann-integrable if:

$$\sup L(f,P) = \inf U(f,P) = I$$

Varying the partition, so

$$I = \int\limits_c^d\int\limits_a^bf(x,y)dxdy$$

\section{Properties}

	\subsection{Differentiability}
	Let $f(x,y)$ be a continuous function from $(a,b)\times (c,d)$ then the function is integrable.
	This result can be extended with different kind of integrals.
	Also this definition can be easily extended in more dimension.

	\subsection{Order of integration}
	If $f(x,y)$ is continuous on $(a,b)\times (c,d)$ then:

	$$\int\limits_c^d\biggl(\int\limits_a^b f(x,y)dx\biggr)dy = \int\limits_a^b\biggl(\int\limits_c^d f(x,y)dy\biggr)dx$$

	\subsection{x-simple and y-simple domains}
	IN some cases the domain of the function is not defined over a rectangular domain.
	One case of easy integration is when the domain is x-simple or y-simple.
	In the case of a y-simple domain, the function is bounded on the $x$ axis by two numerical values and on the $y$ axis by two continuous function $y = g_1(x)$ and $y = g_2(x)$.
	The case of a x-simple domain is the symmetric of the y-simple one.
	Let $f(x,y)$ be a continuous function defined on an x-simple domain $\Omega$:

	$$\Omega = \{(x,y)\in\mathbb{R}, c\le y\le d, h_1(y)\le x\le h_2(y)\}$$

	The integral can be computed as:

	$$\iint\limits_{\Omega} f = \int\limits_c^ddy\biggl(\int\limits_{h_1(y)}^{h_2(y)}f(x,y)dx\biggr)$$

	In the same way for an y-simple domain:

	$$\Omega = \{(x,y)\in\mathbb{R}, g_1(x)\le y\le g_2(x), a\le x\le b\}$$

	$$\iint\limits_\Omega f = \int\limits_a^bdx\biggl(\int\limits_{g_1(x)}^{g_2(x)}f(x,y)dy\biggr)$$

	\subsection{Change of variables}
	The absolute value of the jacobian determinant gives the change in the volume element when passing from a set of coordinate to another.
	In two dimension, given $f(x,y)$ defined on $\Omega$ and supposing to change the integral variables from $x$ and $y$ to $u$ and $v$:

	$$\iint\limits_\Omega f(x,y)dxdy = \iint\limits_{\Omega^*} f(x(u,v), y(u,v))|J|dudv$$

	Where $\Omega^*$ is the new region of integration in the $(u,v)$ plane.
