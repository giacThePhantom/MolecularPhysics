\chapter{Solved exercises}

\section{Complex numbers}

  \subsection{Conversion to polar form}
  Rewrite $\cos(3-2i)$ in polar form.
  \begin{align*}
    \cos(3-2i) & = \frac{e^{i(3-2i)} + e^{-i(3-2i)}}{2}
               & \text{since }\cos\phi = \frac{e^{i\phi} + e^{-i\phi}}{2} \\
               & = \frac{e^{3i}e^{2} + e^{-3i}e^{-2}}{2}
               & \text{distributive property} \\
               & = \frac{(\cos3 + i\sin3)e^{2} + (\cos3 - i\sin3)e^{-2}}{2}
               & \text{since } e^{i\phi} = \cos\phi + i\sin\phi \\
               & = \cos3\left(\frac{e^2+e^{-2}}{2}\right) + i\sin3\left(\frac{e^2-e^{-2}}{2}\right)
               & \text{collecting } \cos3 \text{ and } i\sin3 \\
               & = \underbrace{\cos3\cosh2}_{\text{real part}} + \underbrace{i\sin3\sinh2}_{\text{immaginary part}}
               & \text{since } \cosh x = \frac{e^x+e^{-x}}{2}, \sinh x = \frac{e^x-e^{-x}}{2} \\
  \end{align*}

  \subsection{Multivalued functions}
  Compute $1^{\frac{1}{6}}$, i.e. find all roots of the equation $z^{6} = 1$ in $\mathbb{C}$.
  \begin{align*}
    1^{\frac{1}{6}} & = e^{\frac{1}{6}\ln(1)}
                    & \text{since } a^z = e^{z\ln(a)}\\
                    & = e^{\frac{1}{6}2i\pi k}
                    & \text{since } \ln(re^{i\phi}) = \ln(r) + i\phi + 2i\pi k, 
                                    r = |\sqrt1| = 1, 
                                    \phi = \arctan{0} = 0\\
                    & = e^{\frac{1}{3}i\pi k} \,\,\forall k \in \mathbb{N}^* \wedge k < 6
                    & \text{where } r = 1, \phi' = \frac{1}{3} \pi k
  \end{align*}
  When represented on an Argand plane, these complex numbers are vectors of lenght 1, and the corresponding angles are the ones obtained by substituting the values of k in $\phi'$, namely the set $\{0, \frac{\pi}{3}, \frac{2\pi}{3}, \pi, \frac{4\pi}{3}, \frac{5\pi}{3}\}$.

  Compute $i^i$.
  \begin{align*}
    i^i & = e^{i\ln i}
        & \text{since } a^z = e^{z\ln(a)} \\
        & = e^{i\left(\frac{i\pi}{2} + 2i\pi k\right)}
        & \text{since } \ln(re^{i\phi}) = \ln(r) + i\phi + 2i\pi k, 
                        r = |\sqrt1| = 1, 
                        \phi = \lim_{x \to \infty}\arctan{x} = \frac{\pi}{2} \\
        & = e^{-\frac{\pi}{2}-2\pi k}
        & \text{collecting } i\pi \text{ and simplifying } i^2 = -1 \\
        & = e^{-\frac{\pi}{2}}
        & \text{since } 2\pi = 0 \text{, thus you can simplify the k rotations} \\
        & \approx 0.2079
  \end{align*}
  Notice that $i^i$ is a unique real value, despite being defined using only $i$.

  Compute $1^i$.
  \begin{align*}
    1^i & = e^{i\ln 1}
        & \text{since } a^z = e^{z\ln(a)} \\
        & = e^{i0}
        & \text{since } \ln1 = 0 \\
        & = e^{0} = 1
  \end{align*}
  Notice the fact that "$1$ elevated to any power is equal to $1$" holds even in $\mathbb{C}$.


\section{Partial derivatives}

  \subsection{Solving partial derivatives}
  Compute both first order partial derivatives of $f = e^{ix}(x^3 + y^3 + 1)$.
  \begin{align*}
    \frac{\partial f}{\partial x} & = ie^{ix}(x^3 + y^3 + 1) + e^{ix}(3x^2) \\
                                  & = e^{ix}(ix^3 + iy^3 + i + 3x^2) \\
    \frac{\partial f}{\partial y} & = 3y^2e^{ix} \\
  \end{align*}

  Compute $\frac{\partial^5f}{\partial^2x\partial^3y}$ with $f = xy^3e^{-\frac{1}{y}} + 5e^{ix^3}y^2$.
  \begin{align*}
    f                                           & = xy^3e^{-\frac{1}{y}} + 5e^{ix^3}y^2 \\
    \frac{\partial f}{\partial x}               & = y^3e^{-\frac{1}{y}} + i15x^2e^{ix^3}y^2\\
    \frac{\partial^2f}{\partial^2x}             & = i15y^2[(2x)(e^{ix^3}) + (x^2)(i3x^2e^{ix^3})]\\
                                                & = i15y^2xe^{ix^3}(2 + i3x^3)\\
    \frac{\partial^3f}{\partial^2x\partial y}   & = i30yxe^{ix^3}(2 + i3x^3) \\
    \frac{\partial^4f}{\partial^2x\partial^2y}  & = i30xe^{ix^3}(2 + i3x^3) \\
    \frac{\partial^5f}{\partial^2x\partial^3y}  & = 0 \\
  \end{align*}

\section{Differential operators}
  
  % TODO Add exercises on differential equations (here or section above)

  \subsection{Computing the gradient of a function}
  Compute $\text{grad}(f)$ with $f = x^2 + y^2 + z^2$.
  \begin{align*}
    \text{grad}(f) = \nabla f 
      & = \begin{pmatrix} 
            \frac{\partial f}{\partial x} &
            \frac{\partial f}{\partial y} &
            \frac{\partial f}{\partial z}
          \end{pmatrix} \\
      & = \begin{pmatrix}
            2x & 2y & 2z
          \end{pmatrix}
  \end{align*}
  % TODO Normalize result
  % TODO Use vector notation?

  Compute $\text{grad}(f)$ with $f = \sin(x^2yz)$.
  \begin{align*}
    \text{grad}(f) = \nabla f 
      & = \begin{pmatrix} 
            \frac{\partial f}{\partial x} &
            \frac{\partial f}{\partial y} &
            \frac{\partial f}{\partial z}
          \end{pmatrix} \\
      & = \begin{pmatrix}
            xyz\cos(x^2yx) & 
            x^2z\cos(x^2yx) &
            x^2y\cos(x^2yx)
          \end{pmatrix}
  \end{align*}

  \subsection{Computing the divergence of a function}
  Compute $\text{div}(\vec f \,)$ with $\vec f = \begin{pmatrix} xy & y & zx^2\end{pmatrix}$.
  \begin{align*}
    \text{div}(\vec f \,) = \vec\nabla \cdot \vec f 
      & = 
        \begin{pmatrix} 
          \frac{\partial}{\partial x} &
          \frac{\partial}{\partial y} &
          \frac{\partial}{\partial z}
        \end{pmatrix} \cdot
        \begin{pmatrix}
          f_x & f_y & f_z
        \end{pmatrix} \\
      & = 
        \frac{\partial f_x}{\partial x} +
        \frac{\partial f_y}{\partial y} +
        \frac{\partial f_z}{\partial z} \\
      & = 
        \frac{\partial (xy)}{\partial x} +
        \frac{\partial (y)}{\partial y} +
        \frac{\partial (zx^2)}{\partial z} \\
      & = y + 1 + x^2 \\
  \end{align*}

  \subsection{Computing the curl of a function}
  Compute $\text{rot}(\vec f \,)$ with $\vec f = \begin{pmatrix} y & -x & 0\end{pmatrix}$.
  \begin{align*}
    \text{rot}(\vec f \,) = \vec\nabla \times \vec f 
      & = 
        \begin{pmatrix} 
          \frac{\partial f_z}{\partial y} - \frac{\partial f_y}{\partial z} &
          \frac{\partial f_x}{\partial z} - \frac{\partial f_z}{\partial x} &
          \frac{\partial f_y}{\partial x} - \frac{\partial f_x}{\partial y} 
        \end{pmatrix} \\
      & = 
        \begin{pmatrix} 
          \frac{\partial (0)}{\partial y} - \frac{\partial (-x)}{\partial z} &
          \frac{\partial (y)}{\partial z} - \frac{\partial (0)}{\partial x} &
          \frac{\partial (-x)}{\partial x} - \frac{\partial (y)}{\partial y} 
        \end{pmatrix} \\
      & = 
        \begin{pmatrix}
          0 - 0 & 0 - 0 & -1 - -1 
        \end{pmatrix} \\
      & = 
        \begin{pmatrix}
          0 & 0 & -2
        \end{pmatrix} \\
  \end{align*}
  % TODO Standardize?
  
  \subsection{Curl of the grad of a function}
  Compute $\text{rot}(\text{grad}(\vec f))$.

  \begin{align*}
  \text{rot}(\text{grad}(\vec f \,)) = \vec\nabla \times \left(\vec\nabla \vec f\,\right) 
    & = 
      \vec\nabla \times
      \begin{pmatrix} 
        \frac{\partial f}{\partial x} &
        \frac{\partial f}{\partial y} &
        \frac{\partial f}{\partial z} 
      \end{pmatrix} \\
    & = \det
      \begin{vmatrix} 
        \vec e_x & \vec e_y & \vec e_z \\[6pt]
        \frac{\partial}{\partial x} & 
        \frac{\partial}{\partial y} & 
        \frac{\partial}{\partial z} \\[6pt]
        \frac{\partial f}{\partial x} &
        \frac{\partial f}{\partial y} &
        \frac{\partial f}{\partial z} 
      \end{vmatrix} \\
    & = 
      \begin{pmatrix} 
        \frac{\partial^2 f}{\partial y \partial z} - \frac{\partial^2 f}{\partial z \partial y} &
        \frac{\partial^2 f}{\partial z \partial x} - \frac{\partial^2 f}{\partial x \partial z} &
        \frac{\partial^2 f}{\partial x \partial y} - \frac{\partial^2 f}{\partial y \partial x}
      \end{pmatrix} \\
    & \text{since you can change the order of derivation} \\
    & = 
      \begin{pmatrix}
        0 & 0 & 0
      \end{pmatrix}
  \end{align*}
  Notice that $\text{rot}(\text{grad}(\vec f))$ is equal to the zero vector regardless of $f$.

  \subsection{Divergence of the curl of a function}
  Compute $\text{div}(\text{rot}(\vec f))$.

  \begin{align*}
  \text{div}(\text{rot}(\vec f \,)) = \vec\nabla \cdot \left(\vec\nabla \times \vec f\,\right) 
    & = 
      \vec\nabla \cdot
      \begin{pmatrix} 
        \frac{\partial f_z}{\partial y} - \frac{\partial f_y}{\partial z} &
        \frac{\partial f_x}{\partial z} - \frac{\partial f_z}{\partial x} &
        \frac{\partial f_y}{\partial x} - \frac{\partial f_x}{\partial y} 
      \end{pmatrix} \\
    & = 
      \begin{pmatrix}
        \frac{\partial f}{\partial x} &
        \frac{\partial f}{\partial y} &
        \frac{\partial f}{\partial z}
      \end{pmatrix}
      \cdot
      \begin{pmatrix} 
        \frac{\partial f_z}{\partial y} - \frac{\partial f_y}{\partial z} &
        \frac{\partial f_x}{\partial z} - \frac{\partial f_z}{\partial x} &
        \frac{\partial f_y}{\partial x} - \frac{\partial f_x}{\partial y} 
      \end{pmatrix} \\
    & = 
      \frac{\partial^2 f_z}{\partial y \partial x} - \frac{\partial^2 f_y}{\partial z \partial x} +
      \frac{\partial^2 f_x}{\partial z \partial y} - \frac{\partial^2 f_z}{\partial x \partial y} +
      \frac{\partial^2 f_y}{\partial x \partial z} - \frac{\partial^2 f_x}{\partial y \partial z} \\
    & \text{then, changing the order of the terms we notice that} \\
    & =
      \frac{\partial^2 f_z}{\partial y \partial x} - \frac{\partial^2 f_z}{\partial x \partial y} +
      \frac{\partial^2 f_x}{\partial z \partial y} - \frac{\partial^2 f_x}{\partial y \partial z} +
      \frac{\partial^2 f_y}{\partial x \partial z} - \frac{\partial^2 f_y}{\partial z \partial x} \\
    & \text{since you can change the order of derivation} \\
    & = 0
  \end{align*}
  Notice that $\text{div}(\text{rot}(\vec f))$ is equal to zero vector regardless of $f$.

  \subsection{Divergence of the gradient of a function}
  Demonstrate that $\Delta f = \text{div}(\text{grad}(f))$
  
  \begin{align*}
  \text{div}(\text{grad}(\vec f\,)) = \vec\nabla \cdot \left(\vec\nabla \vec f\,\right) 
    & = 
      \vec\nabla \cdot
      \begin{pmatrix} 
        \frac{\partial f}{\partial x} &
        \frac{\partial f}{\partial y} &
        \frac{\partial f}{\partial z}
      \end{pmatrix} \\
    & =
      \begin{pmatrix} 
        \frac{\partial}{\partial x} &
        \frac{\partial}{\partial y} &
        \frac{\partial}{\partial z}
      \end{pmatrix}
      \cdot
      \begin{pmatrix} 
        \frac{\partial f}{\partial x} &
        \frac{\partial f}{\partial y} &
        \frac{\partial f}{\partial z}
      \end{pmatrix} \\
    & =  
      \frac{\partial^2 f}{\partial^2 x} +
      \frac{\partial^2 f}{\partial^2 y} +
      \frac{\partial^2 f}{\partial^2 z} \\
    & = f
      \left(
      \frac{\partial^2}{\partial^2 x} +
      \frac{\partial^2}{\partial^2 y} +
      \frac{\partial^2}{\partial^2 z}
      \right) \\
    & = \Delta f
  \end{align*}

  
% Placeholders to populate with exercises
% Possibly splitting in multiple files if particularly long
\section{Spherical coordinates}
  % TODO

\section{Multidimensional integrals}
  % TODO

\section{Wave equations}
  % TODO

\section{Hilbert spaces}
  % TODO

  Template to add exercises
  \begin{align*}
    a &
      & \\
  \end{align*}