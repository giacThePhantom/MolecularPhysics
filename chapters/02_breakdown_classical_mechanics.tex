\chapter{Breakdown of classical mechanics}

\section{The fall of determinism}

  \subsection{Double slit experiment}
  Consider a gun shooting particle through a screen with two holes and a detection screen behind the first one.

    \subsubsection{Case 1 - particles are macroscopic bullets}
    If particles are bullets and macroscopic on the second detector a particle-like behaviour is detected.
    Bullets arrive one-by-one.
    First considering the experiment with one of the holes shut a Gaussian like probability of detection $P_1$ can be seen such that $\mu$ is directly perpendicular to the hole.
    If both holes are open there is a ballistic behaviour and the resulting distribution of detection $P_{12}$ is the sum of the two deriving for each hole open by itself:

    $$P_{12} = P_1+P_2$$

    \subsubsection{Case 2 - macroscopic waves in a tank}
    In the case of macroscopic waves in a tank, can be seen that $P_{12} \neq P_1+P_2$.
    This is because wave aptitudes are complex objects.
    So, deriving from wave theory:

    $$A_1\rightarrow P_1 = |A_1|^2 = A_1^*A_1$$
      
    $$A_2\rightarrow P_2 = |A_2|^2 = A_2^*A_2$$

    \begin{align*}
      A_{12} = A_1+A_2\rightarrow A_{12} &= |A_1+A_2|^2=\\
                                         &=A_1^*A_1 + A_2^*A_2 + \underbrace{A_1^*A_2 + A_1A_2^*}_{\text{interference}}
    \end{align*}

    Unlike bullets, wave hit the entire screen and not at a precise time.
    So a wave-like behaviour with de localization can be seen.

    \subsubsection{Case 3 - cathode as an electron gun}
    Finally considering a cathode to be an electron gun.
    If one hole is blocked electrons are detected one by one and have particle-like behaviour.
    If both holes are open quantum delocalization happens and an interference pattern can be seen ($P_{12} \neq P_1 + P_2$) and have wave-like behaviour.
    So detections reveals particle behaviour and the propagation wave-like behaviour in which the electron is everywhere like a wave.
    To see if he electron goes through both holes simultaneously an apparatus that emits a signal if an electron travels nearby is put near the holes.
    Detection on the screen happens only if the signal is emitted.
    The two apparatus never trigger together and the electron travels through one of the holes like a particle.
    Defining $P_{A_1}$ the probability that counts only events in which $A_1$ is triggered and $P_{A_2}$ the results of counting only events triggering $A_2$.
    Counting all the events can be seen that the resulting pattern is $P_{A_1}+P_{A_2}$.
    In this case so $P_{A_1+A_2} = P_{A_1} + P_{A_2}$, delocalization is lost and the electron assumes ballistic behaviour.
    So can be seen that the measurement affects the nature of the electron and can change the state of the system.

    \subsubsection{Conclusions}
    The notion of trajectory looses significance for microscopic particles.
    This is quantified by Heisemberg's uncertainty principle:

    $$\underbrace{\Delta p}_{\text{Uncertainty on }p}\ \underbrace{\Delta q }_{\text{Uncertainty on }q} \ge \frac{\hbar}{2}$$

    It is impossible to simultaneously measure with arbitrary accuracy the position and the velocity of a microscopic partile.

\section{The photoelectric effect}
Considering the classical theory of the hydrogen atom and ignoring that energy loss through electromagnetic radiation would be unstable, this model can transfer any amount of energy to the electron by shining light on it.
In classical electromagnetism the energy of radiation comes from the intensity of the electromagnetic wave.
It would be expected that, irregardless of the frequency or wave-length the amount of electron extracted woud scale with the intensity of the electromagnetic wave.

  \subsection{Experimental findings}
  Electrons are extracted only if the light has a $\nu > \nu_{\min}$ or $\lambda<\lambda_{\max}$.
  If $\nu>\nu_{\min}$ the amount of electrons scale with the intensity.

  \subsection{Conclusions}
  This experiment determined that the energy transfer depends on the frequency $\nu$ of the electromagnetic radiation.
  Moreover electron can only acquire certain quanta of energy.
  This led to the introduction of two pivotal  concepts of modern physics.

    \subsubsection{Energy quantization}
    The amount of energy transfer to a bound system cannot be arbitrary small.

    \subsubsection{Photons}
    Light is made by the photon particle which carries energy $E = \hbar\nu$, where $\omega = 2\pi\nu$ and $\hbar = \frac{h}{2\pi}$, so:

    $$E = \hbar\omega$$

    Photons, like electrons share wave-like and particle-like properties.
    So electrons can be considered as waves of matter and photons as particles of light.

\section{Quantization and atomic spectra}

  \subsection{Experiment}
  A beam of light goes through an atom and a prism.
  The prism splits the frequencies and those frequencies are collected on a screen.

  \subsection{Finding}
  Performing this experiment has been found that only certain frequencies can be absorbed by the atoms.

  \subsection{Conclusion}
  Only certain excitation energies are permitted and those form a characteristic signature of atoms:

  $$\hbar\omega = E_n-E_m$$

  Because of this finding classical mechanics can be seen is not a fundamental theory: it perfectly describes observations in some limited range of length, mass, temperature.
  Any more fundamental theory must contain classical mechanics as an approximation in the macroscopic regime according to the correspondence principle.

\section{Stern Gerlach experiment}

  \subsection{Experiment}
  An electron beam shoots electrons with different momenta $\vec{\mu}$ through an inhomogeneous magnetic field $SG$.
  Behind this there is a screen that detect those electron.

  \subsection{Finding}
  Classically, the electrons are expected to be bended by $SG$ more or less depending on the orientation of $\vec{\mu}$.
  With an expected density shaped like a Gaussian with $\mu$ at the middle.
  Since $\hat{z}$ has been selected by $SG_1$, $SG_3$ would  be expected to find only $\mu_z = + \mu_0$.
  The final beam contains $\mu_z =\pm \mu_0$.

  \subsection{Conclusion}
  The act of measuring $\mu_x$ completely destroy the information about the state of $\mu_z$.
  According to the uncertainty principle $\mu_x$ and $\mu_z$ cannot be simultaneously determined.
  Instead we find only two possible orientation of the magnetic moment that is in fact quantizied.
  Considering three $SGs$ in series with different orientations: $\hat{z}\rightarrow\hat{x}\rightarrow\hat{z}$:
  The first selects $\hat{\mu}_z = +\mu_0$ and the second $\hat{\mu}_x = +\mu_0$.
