\chapter{Breakdown of classical mechanics}

\section{The fall of determinism}

  \subsection{Double slit experiment}
  Consider a gun shooting particle through a screen with two holes and a detection screen behind the first one.

    \subsubsection{Case 1 - particles are macroscopic bullets}
    If particles are bullets and macroscopic on the second detector a particle-like behaviour is detected.
    Bullets arrive one-by-one.
    First considering the experiment with one of the holes shut a Gaussian like probability of detection $P_1$ can be seen such that $\mu$ is directly perpendicular to the hole.
    If both holes are open there is a ballistic behaviour and the resulting distribution of detection $P_{12}$ is the sum of the two deriving for each hole open by itself:

    $$P_{12} = P_1+P_2$$

    \subsubsection{Case 2 - macroscopic waves in a tank}
    In the case of macroscopic waves in a tank, can be seen that $P_{12} \neq P_1+P_2$.
    This is because wave aptitudes are complex objects.
    So, deriving from wave theory:

    $$A_1\rightarrow P_1 = |A_1|^2 = A_1^*A_1$$
      
    $$A_2\rightarrow P_2 = |A_2|^2 = A_2^*A_2$$

    \begin{align*}
      A_{12} = A_1+A_2\rightarrow A_{12} &= |A_1+A_2|^2=\\
                                         &=A_1^*A_1 + A_2^*A_2 + \underbrace{A_1^*A_2 + A_1A_2^*}_{\text{interference}}
    \end{align*}

    Unlike bullets, wave hit the entire screen and not at a precise time.
    So a wave-like behaviour with de localization can be seen.

    \subsubsection{Case 3 - cathode as an electron gun}
    Finally considering a cathode to be an electron gun.

    \subsubsection{Conclusions}

\section{The photoelectric effect}

  \subsection{Experimental findings}

  \subsection{Conclusions}

\section{Quantization and atomic spectra}

  \subsection{Experiment}

  \subsection{Finding}

  \subsection{Conclusion}

\section{Stern Gerlach experiment}

